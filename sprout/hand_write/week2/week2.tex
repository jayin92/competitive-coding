\documentclass[12pt,a4paper]{article}
%加這個就可以設定字體
\usepackage{fontspec}
%使用xeCJK,其他的還有CJK或是xCJK
\usepackage{xeCJK}
\usepackage{enumerate}
\usepackage{amssymb}
\usepackage[margin=2cm]{geometry}

%設定英文字型,不設的話就會使用預設的字型

%設定中英文的字型
%字型的設定可以使用系統內的字型,而不用像以前一樣另外安裝
\setCJKmainfont{Noto Sans CJK TC}

%中文自動換行
\XeTeXlinebreaklocale "zh"

%文字的彈性間距
\XeTeXlinebreakskip = 0pt plus 1pt

%設定段落之間的距離
\setlength{\parskip}{0.3cm}

%設定行距
\linespread{1.5}\selectfont

\title{資訊之芽手寫作業\\第二周}
\author{李杰穎}
\date{}

\begin{document}
\maketitle
\begin{enumerate}
\item 請回答以下問題:
    \begin{enumerate}
        \item $$\lim_{n\to\infty}\frac{3n+1}{n-1} = \lim_{n\to\infty}\frac{3+\frac{1}{n}}{1-\frac{1}{n}} = \frac{3}{1} = 3 $$
        \item $$\lim_{n\to\infty}\frac{n}{n^2+1}=\lim_{n\to\infty}\frac{\frac{1}{n}}{1+\frac{1}{n^2}}=\frac{0}{1+0}=0$$
        \item% 
            $$\textrm{(1) 先證明}f(n)\in O(2^n)\Rightarrow f(n)\in O(2^{n+1})$$    
            $$\because f(n) \in O(2^n)$$
            $$\therefore \forall k \geq 0, \lim_{n\to\infty}\frac{f(n)}{2^n}=k$$
            $$\therefore \lim_{n\to\infty}\frac{f(n)}{2^{n+1}}=\lim_{n\to\infty}\frac{f(n)}{2^n}\cdot\lim_{n\to\infty}\frac{2^n}{2^{n+1}}=k\cdot \frac{1}{2}=\frac{1}{2}k \textrm{(仍為常數)}$$
            \begin{equation}
                \therefore f(n) \in O(2^{n+1})
            \end{equation}
            $$\textrm{(2) 再證明}f(n)\in O(2^{n+1})\Rightarrow O(2^n)$$
            $$\because f(n) \in O(2^{n+1})$$
            $$\therefore \forall k \geq 0, \lim_{n\to\infty}\frac{f(n)}{2^{n+1}}=k$$
            $$\therefore \lim_{n\to\infty}\frac{f(n)}{2^n}=\lim_{n\to\infty}\frac{f(n)}{2^n}\cdot\lim_{n\to\infty}\frac{2^{n+1}}{2^n}=2k\textrm{(仍為常數)}$$
            \begin{equation}
                \therefore f(n) \in O(2^{n})
            \end{equation}
            $$\therefore \textrm{由}(1), (2)\textrm{得證}$$
        \item%
            令$f(n)=(n+1)!$,則:
            $$\therefore \lim_{n\to\infty}\frac{f(n)}{n!}=\lim_{n\to\infty}\frac{(n+1)!}{n!}=\lim_{n\to\infty}n+1$$
            並非收斂為常數,故$f(n)\in O((n+1)!)\Rightarrow f(n) \in O(n!)$不成立。\\
            故原命題不成立。
        \item
    \end{enumerate}
    
\end{enumerate}
\end{document}