\documentclass[12pt,a4paper]{article}
%加這個就可以設定字體
\usepackage{fontspec}
%使用xeCJK,其他的還有CJK或是xCJK
\usepackage{xeCJK}
\usepackage{enumerate}
\usepackage{amssymb}
\usepackage[margin=2cm]{geometry}

%設定英文字型,不設的話就會使用預設的字型

%設定中英文的字型
%字型的設定可以使用系統內的字型,而不用像以前一樣另外安裝
\setCJKmainfont{Noto Sans CJK TC}

%中文自動換行
\XeTeXlinebreaklocale "zh"

%文字的彈性間距
\XeTeXlinebreakskip = 0pt plus 1pt

%設定段落之間的距離
\setlength{\parskip}{0.3cm}

%設定行距
\linespread{1.5}\selectfont

\title{資訊之芽手寫作業\\第一周}
\author{李杰穎}
\date{}

\begin{document}
\maketitle
\section{第一題}
我們首先假設$n$個人中沒有人不認識其他人,否則其可以視為$n-1$的情況。\\
已知一人最多認識$n-1$個人,但現在總共有$n$個人。\\
故根據鴿籠原理可知至少有兩個人的認識人數相同。\\
\section{第二題}
\begin{enumerate}
    \item  當$n=3$時,$3^3 + 4^3 < 5^3$成立
    \item 令$n=k$時,$3^k + 4^k < 5^k$成立
    \item 當$n=k+1$時:
        $$
        3^{k+1} + 4^{k+1} = 3 \cdot 3^k + 4 \cdot 4^k < 4 \cdot 3^k + 4 \cdot 4^k = 4 \cdot (3^k+4^k) < 4 \cdot 5^k <5 \cdot 5^k = 5^{k+1}
        $$
    故得證當$n \geq 3,3^n + 4^n < 5^n$
\end{enumerate}
\section{第四題}
\begin{enumerate}
    
    \item 當$n=1$時,序列$s=\{1\}$,得分總和為$0$,故$\frac{n(n-1)}{2}=0$成立
    \item 令$n=k$時,得分總和=$\frac{k(k-1)}{2}$成立
    \item 當$n=k+1$時:
        序列$s=\{k+1\}$,我們可以將$k+1$分成$k$及$1$,此分數為$k \cdot 1 = k$。\\
        已知$k$的操作總和=$\frac{k(k-1)}{2}$,故當$n=k+1$時,其得分總和為:\\
        $$\frac{k(k-1)}{2}+k=\frac{k^2-k+2k}{2} = \frac{k^2+k}{2} = \frac{k(k+1)}{2} = \frac{(k+1)(k+1-1)}{2}$$
        故得證。
        
\end{enumerate}

\end{document}